\documentclass{article}
\usepackage[utf8]{inputenc}
\usepackage{amsmath,amssymb,amsthm,pxfonts,listings,color}
\usepackage[colorlinks]{hyperref}
\definecolor{gray}{rgb}{0.6,0.6,0.6}

\usepackage{caption}
\DeclareCaptionFormat{listing}{\llap{\color{gray}#1\hspace{10pt}}\tt{}#3}
\captionsetup[lstlisting]{format=listing, singlelinecheck=false, margin=0pt, font={bf}}

\lstset{columns=fixed,basicstyle={\tt},numbers=left,firstnumber=auto,basewidth=0.5em,showstringspaces=false,numberstyle={\color{gray}\scriptsize}}

\newcommand{\Ref}[2]{\hyperref[#2]{#1 \ref*{#2}}}

% Shamelessly swiped from
% http://compgroups.net/comp.text.tex/using-ref-and-label-so-that-ref-points-to-the-l/245066
\makeatletter
\newcommand*{\Label}[2]{%
  \@bsphack
  \begingroup
    \label{#1-original}%
    \def\@currentlabel{#2}%
    \label{#1}%
  \endgroup
  \@esphack
}
\makeatother


\lstnewenvironment{asmcode}       {}{}
\lstnewenvironment{cppcode}       {\lstset{language=c++}}{}
\lstnewenvironment{javacode}      {\lstset{language=java}}{}
\lstnewenvironment{javascriptcode}{}{}
\lstnewenvironment{htmlcode}      {\lstset{language=html}}{}
\lstnewenvironment{perlcode}      {\lstset{language=perl}}{}
\lstnewenvironment{rubycode}      {\lstset{language=ruby}}{}
\lstnewenvironment{pythoncode}    {\lstset{language=python}}{}

\lstnewenvironment{resourcecode}{}{}

\title{Interviewing in Ten Minutes}
\author{Spencer Tipping}

\begin{document}
\maketitle{}
\tableofcontents{}

\section*{About this guide}
Over the past few years I've worked for a lot of startups. (Because I
job-hopped, not because I'm important.) For one reason or another, I ended up
conducting technical interviews at about half of these companies. Usually we
were looking for developers who knew some high-end functional language, so
this guide is strongly biased by that -- though I think it also applies to
more standard startup jobs.

{\bf Everything in this guide is my own opinion.} It does not reflect the
views of any of my employers, past or present.

\section{The business protocol}\label{sec:business-protocol}
  For various reasons, partly tradition and partly expedience, companies tend
  to have a set of norms that dictate how people work together. From what I've
  been able to observe, the following commandments encode the business
  protocol:

\begin{enumerate}
\item{\bf Thou shalt respect other people's time like thine own.}
  This is absolutely crucial. Business is about money, people convert money
  to time, and if you waste/take an employee's time then you're wasting the
  company's money. People holding the money to pay your salary want to see
  a positive return on their investment in you, which is undermined if you
  then make other people less productive. {\bf Be the person who makes
  others more productive.}

\item{\bf Thou shalt get things done.}
  By this I mean, if someone gives you a task it will be done. They don't
  have to think twice about it. If you hit difficulty, you'll figure it
  out. If you can prove it's impossible, change the problem to be possible
  and explain why you did. {\bf Be the person who gives up last.}

\item{\bf Thou shalt work only when it is productive.}
  Work $\ne$ productivity. The two are nearly orthogonal, in fact.
  Understand what is productive at as high a level as you can, and work on
  that stuff. Busywork is worse than doing nothing because you'll be
  unavailable but will be getting nothing done. {\bf Be the person who does
  everything for a reason.}

\item{\bf Thou shalt simplify.}
  This is a way to respect people's time. Every problem that comes your way
  should be simpler after it interacts with you. Make things look easy
  because to you they are easy. {\bf Be the person who finds the best
  solution.}

\item{\bf Thou shalt be fluent.}
  This goes two ways:~you must communicate flawlessly both with people and
  with computers. Functionally, a programmer is an information channel that
  translates between human intent and computational action. You want this
  channel to be efficient, lossless, and high-capacity. {\bf Be the best
  conduit of intent from people to machines.}

\item{\bf Thou shalt not bikeshed.}
  Thou shalt not argue in general, in fact. But if you must, make sure it's
  about an issue that justifies a conversation costing the company more
  than \$100/hour (nearly every conversation is likely to cost this, and
  some will cost far more). {\bf Be the person who cuts through the noise
  and focuses on real problems.}
\end{enumerate}

  I'll go into more details about how to convey these things during an
  interview later on. For now, though, understand that the above values (or
  similar ones) serve as the conduit by which your skills can be applied to the
  company's problems. If this conduit is missing or sufficiently damaged, your
  skills won't matter.

\section{The interviewer}\label{sec:the-interviewer}
  Interviewers are not usually sadists. While the case could be made that the
  job demands a certain degree of malevolence, most of the confrontation during
  the interview is due to the economics underlying hiring.

  Consider the common flow of resume screen, phone screen, and then in-person
  interviews. There's a progression of risk/cost associated with moving forward
  at each stage:~screening a resume requires minutes, a phone interview
  requires an hour, and an in-person interview requires several people-hours,
  possibly with hotel and airfare. In dollar amounts, these could easily
  represent investments of \$10, \$100, and \$1000.\footnote{I've been told
  that in-person interviews can cost significantly upwards of \$1000, so take
  these numbers with a grain of salt and treat them as lower bounds.} The worst
  case, a bad hire, costs potentially \$10,000 or more.

  Given this, the interviewer operates as an investor of the company's
  assets.\footnote{Unlike most investors, however, their pay is constant; so
  they don't have a stake in the outcome of the interview. This creates a
  slight negative bias, but as far as I know it shouldn't influence their
  decisions otherwise.} Their job is to evaluate risks for the company by
  making bets on people. Because of this, if you are an interviewee, your job
  is to sell such a bet to the person interviewing you. Perhaps more to the
  point, the interviewer is likely to be conservative because they don't get
  paid commission; so your goal is to be so awesome that they feel safer saying
  yes than no.

\subsection{Aside:~interview attire}
  {\bf The interviewer does not care what you wear to an interview.} Nobody
  cares about this except you. Unlike in almost any other industry, a suit is
  much more likely to be a liability than anything else. I have gotten every
  job I've had by going to interviews in flip-flops, khakis, and a T-shirt, and
  being very comfortable because this is what I'm used to wearing. The
  interview is about your mind, so give your mind the least-distracting
  atmosphere possible so it can do its job.

  It makes perfect sense if you think about the interviewer's position. If
  you're awesome, the interviewer cannot go back to the hiring manager with an
  assessment like, ``well, the candidate is the most productive we've ever
  seen, but can we really put up with someone who dresses that way?''
  ``Fashion-aware'' and ``software developer'' don't even belong in the same
  sentence. The interviewer will probably say something like, ``the candidate
  blew us away on the coding tests, {\em and this one even wore pants!}''
  ``Capable {\em and} pants? Hire.''

  The flip side is equally clear. If you're awkward and uncomfortable, the
  interviewer doesn't know whether it's because you're having trouble with the
  problems, or whether it's that you're wearing a suit to an office with a beer
  pong table and a CEO dressed in a hoodie. The interviewer is certainly not
  going to say, ``well, that candidate can't code his way out of paper bag but
  man is he a sharp dresser. I'm a yes.''

\section{The r\'esum\'e and cover letter}\label{sec:resume}
  At \$100, a phone screen is still fairly cheap. The stakes of a resume and
  cover letter are not usually very high, nor is there very much signal for the
  interviewer to work with. However, if you're about optimizing stuff (and
  what engineer isn't?), there are a few things you can do to communicate at
  this point:

\begin{enumerate}
\item{\TeX{} your resume. There's only upside here, and it's a nice}
  positive signal that you know what you're doing. If you don't know
  \TeX{}, use HTML and export as PDF. If all else fails, write one in plain
  text and claim that minimalism is a social virtue you take seriously.
  Don't use Word because it conveys that you have no standards. As a
  colleague of mine once said, ``if I'm on the fence and the resume is a
  word doc, I typically reject the candidate.'' {\bf Use a spellchecker.}

\item{Keep your resume down to one page. This shows that you respect their}
  time. {\bf Use a spellchecker.}

\item{Write a purposeful but informal cover letter. Be personable,}
  reasonably enthusiastic, and talk about how the stuff you do for fun is
  relevant to the company's hard problems. {\bf Use a spellchecker.}

\item{Link to your github profile. If you don't have one yet, create one}
  and upload something. If you don't have anything to upload, start solving
  Project Euler and upload that. {\bf Use a spellchecker.}

\item{Put some code in your cover letter. Nothing elaborate, but just}
  something you thought was cool, or maybe a challenge problem they have on
  their website, or something. Do this only if it makes sense, but it's a
  great opportunity if you can. {\bf Use a spellchecker.}

\item{Link to a project you worked on that really describes who you are.}
  {\bf Use a spellchecker.}
\end{enumerate}

  There are a few things you absolutely cannot do:

\begin{enumerate}
\item{Talk about salary in any way. It's just too early and it sends the}
  wrong message. It would be like someone asking about your finances before
  the first date. This is a question you ask when you're considering
  marriage, not dinner.

\item{Lie. If you're called out on it, you're done for and people will}
  hate you. Word might get around to other companies if you're unlucky. If
  the truth isn't good enough, fix the truth.

\item{Be noticeably unprofessional. Startups aren't overwhelmingly formal}
  places, but they're just as vulnerable to harrassment lawsuits as anyone
  else. You do not want to give anyone reason to believe that you're going
  to cause legal problems or alienate people.

\item{Bring politics or religion to the table. These topics are emotional}
  catalysts with no upside and enormous potential for problems. Brendan
  Eich, the inventor of Javascript, was forced to resign as CEO of Mozilla
  because he had made a \$1000 contribution against gay marriage six years
  prior. If you have public political statements on a blog or elsewhere,
  take them down.
\end{enumerate}

\section{The technical phone interview}\label{sec:technical-phone-interview}
  At this point you have cost the company \$110 or so, and they're deciding
  whether or not to bump that figure by a factor of 10. This is real money, so
  the burden is on you to convince the interviewer (who may or may not be good
  at interviewing people) that you're the best bet they'll make this year.
  Depending on the job you're applying for, there are a few different ways to
  do that.

  In this section I'm going to assume the technical interview is just talking
  on the phone. The next section is entirely devoted to how to handle a
  collaborative coding interview.

\subsection{Technical literacy}
    This is an absolute must. At a minimum, you need to be able to answer
    questions like these with little or no thought:

\begin{itemize}
\item{Write the FizzBuzz function.}
\item{Reverse an array in place.}
\item{Reverse a linked list in linear time.}
\item{What is the insertion time for a binary tree?}
\item{What is the insertion time for a hashtable (worst, average)?}
\item{How is a priority queue implemented?}
\item{Quicksort vs mergesort, pros/cons?}
\item{What does $O(n^2)$ mean formally?}
\item{What does ``atomic'' mean?}
\item{What is memoization?}
\item{Why do floating-point computations often use an epsilon?}
\item{How are negative integers represented?}
\item{Write the Fibonacci function without using recursion.}
\item{What's the difference between the stack and the heap?}
\item{Conceptually, what does a garbage collector do?}
\item{Conceptually, how does virtual memory work?}
\item{What happens under the hood when you go to {\tt google.com}?}
\item{Why should you profile before optimizing?}
\item{What is an example use case for a cache?}
\item{How would you do [some simple task] in a UNIX shell?}
\end{itemize}

\subsection{Development literacy}
    This is also important, especially if you're applying for a position that
    requires experience. You should be able to speak intelligently about things
    like:

\begin{itemize}
\item{The bugs you most commonly write, and habits you've developed to}
  avoid them.
\item{How expensive it is to fix a bug in production.}
\item{Strengths and weaknesses of unit testing.}
\item{The drawbacks of using powerful languages like Lisp. (You can}
  extol their virtues too, but if that's all you do then people are
  unlikely to take you very seriously.)
\item{Your preferences about design and documentation, and why you do it}
  this way.
\item{A project you've worked on that went really well, and why this}
  happened.
\item{A project you've worked on that was a disaster, and why. Don't try}
  to blame other people; you need one where you messed up and learned
  something. If you don't have a project like this, think of something
  really hard, try to solve it, push the failed results to Github, and
  talk about why that didn't work. It's fine if failure was inevitable;
  the main thing is to know how to translate it into improvement.
\end{itemize}

\subsection{Career/self literacy}
    This is less important from a technical perspective, but it still sends a
    signal and is worth getting right. You should have a reasonably
    well-informed opinion about:

\begin{itemize}
\item{Your preferred development style/environment}
\item{How you like to interact with your manager}
\item{How autonomous you prefer to be (more is not always better)}
\item{The types of projects you like to work on, and why}
\item{Roles that don't suit you}
\item{How you envision you might be useful to this company}
\item{How you want to change over the coming months/years}
\item{What it takes to keep you at a job}
\item{What would make you leave a job}
\item{What would make you leave a job and write a blog post about why}
  you did
\end{itemize}

    This is the part where you are interviewing the company. Just because they
    offer you a job doesn't mean you should take it. Taking the wrong job is
    not only bad for personal growth, but will most likely lead to a minefield
    of a resume that will hurt your future career prospects. Demonstrating
    reliability and the ability to stick it out is important in the long run
    for two reasons. One is that hiring is expensive, so having to replace you
    quickly is a liability. But the other is that people who tend to job-hop
    might be doing it because they're bad at working with people. You want to
    eliminate this possibility in the mind of your interviewer.

\section{Coding interview}\label{sec:coding-interview}
  People tend to assume there are two outcomes for a coding interview:~succeed
  or fail. This is not technically true, however; a third outcome is {\em
  succeed specatularly}, which is what you should do if you have the option,
  and here's why.

  Coding challenge interviewers are usually engineers, and engineers tend to
  like to have answers to questions so they can move on. Because of this,
  they'll try to jump to a conclusion about you as soon as they have enough
  signal to justify it. This conclusion is rarely final, but at least they have
  something to work with if they don't observe anything else.

  The interview question will usually fall into one of two categories, each
  equally expedient. One is stuff like FizzBuzz, which rapidly detects certain
  failure modes. The other is something open-ended that gives you some room to
  be creative. Although these problems can detect failure, they're more often
  about detecting excellence.

  If at any point in the interview process someone decides you're excellent,
  you have a much higher chance of getting an offer. Picasso was excellent and
  his paintings are now worth millions, whereas the guy who paints imitation
  art just like Picasso is broke. Setting a new bar instantly promotes you into
  the class of people who can't be replaced, and this makes it much easier to
  justify further investment in interviewing you.

\subsection{Coding challenge etiquette}
    Enjoy the coding challenge. Relish it. This is an opportunity to be asked
    difficult, interesting questions that are designed to be solvable in less
    than an hour. It's break from all the mundane aspects of writing production
    code, and an opportunity to prove to yourself as well as your interviewer
    that you can solve anything that comes your way.

\begin{itemize}
\item{Have a language you're really comfortable with. Java is not a bad}
  choice for this because it's straightforward and everyone knows it.

\item{Unless the problem is trivial, verbally sketch out your thought}
  process. Argue against the approaches you've taken to see if they hold
  up. Don't write any code until you know what you're doing. The goal of
  all this is to demonstrate that your thought process is effective at
  solving problems. Silence followed by perfection also works, but is
  riskier.

\item{Make simplifying assumptions as necessary, but be prepared to undo}
  them later. Always mention when you're doing this.

\item{If the problem is unsolvable as stated, change it minimally to}
  make it solvable and explain why you need to do this.

\item{If the problem is too vague or ill-specified to even work with,}
  talk with the interviewer to work the problem into something you can
  solve. Everyone loses if you walk away from the interview without
  having done anything.

\item{When writing code, use idioms that reflect a considered}
  perspective. It should be simple, easy to maintain, well-suited to the
  problem, and generally self-documenting. You should be able to
  intelligently discuss every aspect of your coding style.

\item{Never sweat the small stuff. Nobody cares if you know what the}
  deletion method is called on Java's {\tt HashSet} class. Just casually
  mention that you'd look it up.

\item{Never get defensive. If you've made a mistake, fix it quickly (or}
  think about it and discuss it) and move on. Any good interviewer will
  understand that people make mistakes, and some will even set up
  questions specifically designed to mislead you. This is all part of the
  interview and it's to see how you take it when things go wrong.
\end{itemize}

\subsection{Quick-fail problems}
    These are the easiest problems you'll get, so you need to make them look
    easy. Some classic examples:

\begin{enumerate}
\item{Write the factorial function.}
\item{Write FizzBuzz.}
\item{Calculate the sum of numbers in a linked list.}
\item{Print the last line of a file (naive solution).}
\end{enumerate}

    Coding style matters, as well as speed and whether you can write them
    without any bugs. I'll go through some solutions of varying quality.

\subsubsection{FizzBuzz}
      Bad solutions indicate a deep lack of understanding of some sort. Any
      competent interviewer will count a solution like this against you:

\lstset{caption={examples/fizzbuzz-bad.java},name={examples/fizzbuzz-bad.java}}\begin{javacode}
pulic int fizz_buzz() {         // "pulic" -- eh; "int" ... ????
int count = 0;                  // count of what?; also, indent!
while(true) {                   // forever? why?
int count2 = count              // "count2"?
while (count2 > 0) {            // modulus! modulus!
count2 = count2 - 3;
}
int count3 = count;             // ok, this is just silly
while (count3 > 0) {
count3 = count3 - 5;
}
//check for divisibility
if (count2 == 0){
if ( count3 == 0) {
System.out.println("fizzbuzz");
}
else                            // missing open brace, not good
System.out.println ("fizz");
}
else
if count3 == 0 {
System.out.println("buzz")
}
else {
System.out.println(count)
}
}
} \end{javacode}

      This, on the other hand, is a solution I would consider to be nearly
      perfect:

\lstset{caption={examples/fizzbuzz-good.java},name={examples/fizzbuzz-good.java}}\begin{javacode}
for (int i = 1; i < 100; ++i) {
  final boolean by3 = i % 3 == 0;
  final boolean by5 = i % 5 == 0;
  if (by3 && by5)
    println("fizzbuzz");
  else if (by5)
    println("buzz");
  else if (by3)
    println("fizz");
  else
    println(i);
} \end{javacode}

      Some important differences:

\begin{itemize}
\item{Indentation and good style should be habit for you. Code should}
  have meaning, not just behavior. Formatting should be absolutely
  consistent (any random-looking inconsistency such as in the bad
  solution above will lead your interviewer to suspect you might be
  junior at writing code, and you really want to avoid this).

\item{Know the nuances and idioms of your language. Java gives you a}
  modulus operator, so you should use it here. Use the appropriate
  iteration and decisional constructs.

\item{Draw attention to stuff that matters. Here, most of the space is}
  taken up by the condition structure, which is parameterized by two
  variables calculated above. The {\tt for} loop is appropriately
  minimal because it's a standard iteration form that uses a generic
  variable name.

\item{Write the solution to resemble the problem. Here, {\tt by3} and}
  {\tt by5} are the two conditions we care about. This makes the {\tt
  if} stack nearly a verbatim translation of the problem statement.
  Because of this, it's very easy to look at the code and be confident
  it does the right thing. These variables are short, descriptive, and
  not generic, which causes them to draw attention without taking up
  much space.

\item{Write hints into your code (as appropriate). I like to use}
  {\tt final} to indicate invariance, for example. This is a matter of
  opinion, but anything you can do to make it clear that you're
  thinking about maintainability will work in your favor.
\end{itemize}

\subsubsection{Sum of linked list}
      This problem is nice because it has a mediocre solution:

\lstset{caption={examples/sumlist-mediocre.java},name={examples/sumlist-mediocre.java}}\begin{javacode}
interface OurList {
  int head();
  OurList tail();
}
int sum(OurList list) {
  if (list == null)
    return 0;
  else
    return list.head() + sum(list.tail());
} \end{javacode}

      This is logically correct, terse, and generally well-expressed. But it
      has two big problems. First, it returns an {\tt int} rather than a {\tt
      long} or {\tt double}; this is probably a mistake. Second, and more
      importantly, it requires linear stack space and may cause an overflow.
      Java does not optimize tail calls, but even if it did the recursive call
      here is not in tail position. A much better solution uses iteration:

\lstset{caption={examples/sumlist-good.java},name={examples/sumlist-good.java}}\begin{javacode}
interface OurList {
  int head();
  OurList tail();
}
long sum(OurList xs) {
  long total = 0;
  for (; xs != null; xs = xs.tail())
    total += xs.head();
  return total;
} \end{javacode}

      Not only is this more reliable, but it's also quite a bit faster and
      follows normal Java idioms more closely. It also creates some
      opportunities to talk about some edge cases. For example, what if the
      list is dynamically generated and very long? The iterative solution is
      ideal here because it loses references to cells it's already processed,
      which allows them to be garbage collected. This means we can
      theoretically process lists of arbitrary length using constant space.

      I also prefer the parameter name {\tt xs} to a word like {\tt list}. {\tt
      xs}, which is prounced, ``exes,'' seems simpler to me than {\tt list}
      does because it refers to a series of things called {\tt x}, and {\tt x}
      is the most generic name for a data value. It also takes up less space,
      which makes it easier to focus on the algorithm structure.

      The {\tt for} loop is a little nonstandard, but the idiom is preserved
      from Java's usual numeric iteration. This means that although someone
      unaccustomed to this type of loop might be confused at first, the usual
      intuition about {\tt for} loops would all apply:~we're moving forward
      through a list until we hit the end, the termination check is the second
      thing, and the increment is the third.

\subsubsection{Practice problems}
      Solutions to these are at \ref{sec:practice-problem-solutions}.

\begin{enumerate}
\item{$f(s) = \textrm{true}$ if $s$ is a palindrome. Optimal solution}
  is constant space, linear time, 6 lines total. Should take under a
  minute.
\item{$f(xs) = \textrm{true}$ if $xs$ contains a geometric series.}
  Solution is constant space, linear time, 15 lines of code. Should
  take less than 5 minutes.
\item{Write a function that counts the words in a string without}
  allocating memory. Solution is linear time, 14 lines of code, and
  should take less than 5 minutes.
\end{enumerate}

\subsection{Algorithm problems}
    These usually have a known-best solution, but it's ok not to find it. You
    should of course try, and brush up on all of the usual algorithm and data
    structure stuff before a coding interview in case you get a problem like
    this. Below I go through some example problems. It's been long enough since
    I've seen these that I don't remember the solution.

    By the way, you'll notice the almost complete absence of comments from my
    example code. Most interview problems are simple enough that comments are
    unnecessary. When you're practicing questions like this, write your code
    without comments or blank lines and see how readable you can make it. It's
    ok if you later decide to use comments, but most really good code won't
    require them.

\subsubsection{Maximum subarray}
\begin{quote}
\em
Given a list of integers, find the offset and length of the subarray
with the largest sum.
\end{quote}

\paragraph{Initial train of thought}
        It isn't immediately obvious how to do this, so start by talking
        through a few strategies. One option is to calculate the sum of every
        subarray and find the maximum of those, but it seems slow. These
        subarrays are always made from adjacent elements, though, so maybe we
        can ask something like, ``should I expand out by one.'' In fact, since
        we're just summing stuff, we can expand out anytime we have something
        nonnegative and we'll get a larger, or at least not smaller, sum.

        If I were to start coding now (which would be a mistake), here's what I
        would write:

\lstset{caption={examples/max-subarray1.java},name={examples/max-subarray1.java}}\begin{javacode}
// return [offset, length]
int[] subarray(final int[] xs) {
  int nextUntried  = 1;
  int maxSum       = 0;
  int maxSumStart  = -1;
  int maxSumLength = 0;
  for (int i = 1; i < xs.length; i = nextUntried) {
    maxSumStart  = -1;
    maxSumLength = 1;
    while (i + maxSumStart >= 0 && xs[i + maxSumStart] >= 0) {
      --maxSumStart;
      ++maxSumLength;
    }
    i += maxSumStart;
    while (i + maxSumLength < xs.length
           && xs[i + maxSumLength] >= 0)
      ++maxSumLength;
    nextUntried = i + maxSumLength + 1;
  }
  return new int[] {maxSumStart, maxSumLength};
} \end{javacode}

        The code is a little awkward, but it isn't terrible. The problem is
        with my thinking:~it's obvious I don't really understand something
        important about the problem. The right move is to continue thinking
        about it.

\paragraph{First course correction}
        Ok, so what happens when we're expanding and hit a negative number? I
        guess we need to look on the other side to see if there's an even
        larger positive number. That's kind of gross, and bumps us back into
        quadratic time; so let's think about it differently. Suppose we first
        take a running sum of the array; then what are we looking for? Oh!
        We're just finding the max and min of that. The numbers in between
        those extremes collectively have the largest sum.

        Writing code here is a much less serious mistake, but is still
        suboptimal. Here's what I'd say, most likely:

\lstset{caption={examples/max-subarray2.java},name={examples/max-subarray2.java}}\begin{javacode}
int[] subarray(int[] xs) {
  // modify xs in place (!)
  for (int i = 0, total = 0; i < xs.length; ++i)
    xs[i] = total += xs[i];

  int max = 0;
  int min = 0;
  for (int i = 1; i < xs.length; ++i) {
    if (xs[i] > xs[max]) max = i;
    if (xs[i] < xs[min]) min = i;
  }
  return new int[] {Math.min(min, max),
                    Math.max(min, max) - Math.min(min, max)};
} \end{javacode}

        This is much better, but it's wrong and still suboptimal. Let's
        continue thinking:

\paragraph{Second course correction}
        Ok, so how do we do this in code? We could start by summing the array
        in a loop, then check for max and min. Oh, but what if min happens
        after max? I guess we really need the largest min/max gap where the max
        is after the min.

        Do we actually need to sum the array in a loop? We never look backwards
        at stuff we've summed, so really we just need to go through and
        maintain a running total. This is an optimal solution, so coding now
        makes sense:

\lstset{caption={examples/max-subarray3.java},name={examples/max-subarray3.java}}\begin{javacode}
int[] subarray(final int[] xs) {
  if (xs.length == 0)
    return null;
  int max      = 0;
  int min      = 0;
  int maxValue = xs[0];
  int minValue = xs[0];
  for (int i = 1, sum = xs[0]; i < xs.length; ++i) {
    sum += xs[i];
    if (sum < minValue) {
      minValue = maxValue = sum;
      min      = max      = i;
    }
    if (sum > maxValue) {
      maxValue = sum;
      max      = i;
    }
  }
  return new int[] {min, max - min};
} \end{javacode}

        This solution feels a little more complicated than the last one, but
        it's conceptually more straightforward (since we're not modifying
        anything) and better-engineered. Just the fact that we're solving it in
        constant space is a consideration that a good interviewer would
        appreciate.

\subsubsection{Parse simple math}
\begin{quote}
\em
Write a function that takes a string containing a simple expression and
returns the result. The expression will contain single-digit numbers,
{\tt +}, and {\tt *}, and should respect precedence.
\end{quote}

      Ok, this is obviously a shunting-yard style problem. We could parse it
      recursive-descent, but that's overkill. All operators are
      left-associative, so we can evaluate as we go. Actually, we can just have
      two constant levels:~the number on the left of a {\tt +} and the current
      product. So our solution is constant-space.

\lstset{caption={examples/parse-math.java},name={examples/parse-math.java}}\begin{javacode}
double parsemath(final String expr) {
  double left  = 0;
  double right = 1;
  for (int i = 0; i < expr.length(); ++i) {
    final char c = expr.charAt(i);
    switch (c) {
      case '+':
        left += right;
        right = 1;
        break;
      case '*':
        break;
      default:
        right *= c - '0';
        break;
    }
  }
  return left + right;
} \end{javacode}

      I had some doubts about 2/3 of the way through this. I wasn't sure it
      would be ok to add {\tt right} to {\tt left} at the end, since it might
      contain a stray {\tt 1} from either initialization or the {\tt +} case.
      But after some thought I realized it would be ok since we always end
      after seeing a digit. That digit changes the 1 into something meaningful.

      In hindsight, my variable names could be better. {\tt left} and {\tt
      right} are less descriptive than {\tt sum} and {\tt product}. Also, the
      first time through I had forgotten parens after {\tt expr.length}. This
      isn't terrible, but it isn't great either. In my case I was confusing
      Java and Javascript, but an interviewer might assume I was bad at Java
      and confusing strings and arrays.

      One nice part of this solution is \verb|c - '0'|. Using the equivalent
      \verb|c - 0x30| or \verb|c - 48| would be hostile to someone reading your
      code, whereas \verb|c - '0'| gives someone just enough hints that it's
      clear you're not up to anything too complicated.

      Doing things like character-to-number conversion this way tells the
      interviewer you know how characters work and have at least some knowledge
      of ASCII. It's also reminiscent of C-style code, which will usually work
      in your favor in today's hiring culture of very high-level languages and
      app development (it shows perspective, assuming you can also do the
      high-level stuff).

\subsubsection{Sorted word index}
\begin{quote}
\em
Build the index for a book, sorted by word. Your input is a list of
$\langle \textrm{word}, \textrm{page} \rangle$ tuples, and the list may
contain duplicates.
\end{quote}

      We'll need some data structures for this. I'll go ahead and write this up
      the wrong way to illustrate some common mistakes:

\lstset{caption={examples/sorted-index-bad.java},name={examples/sorted-index-bad.java}}\begin{javacode}
List index(List tuples) {
  HashMap resultIndex = new HashMap();
  for (int i = 0; i < tuples.size(); ++i) {
    if (!resultIndex.contains(tuples[i].word))
      resultIndex.add(tuples[i].word, new LinkedList());
    resultIndex.get(tuples[i].word).add(tuples[i].page);
  }
  List sortedResult = new ArrayList();
  Iterator iterator = resultIndex.iterator();
  while (iterator.hasNext()) {
    Tuple indexEntry = iterator.next();
    sortedResult.add(indexEntry);
  }
  quicksort(sortedResult);
  return sortedResult;
} \end{javacode}

      This solution has a lot of problems. For one thing, none of the data
      structures are parameterized. {\em Always parameterize your data
      structures.} There are two reasons you want to do this. First, prior to
      Java 1.5 (a long time ago), Java didn't even have generics. You might
      come across as a Luddite who prefers inferior technologies.

      But more importantly, writing this stuff down tells the interviewer what
      you're thinking. A map of \verb|<Thing, Thing>| doesn't convey your
      intentions very well; its meaning is then defined by what you do to the
      map instead of something intrinsic to it.\footnote{Generics aren't
      intrinsic really, but you get the idea.} Whenever possible, use code to
      document what you're doing. Especially parameters and return types,
      because this gives the interviewer a quick way to correct any
      inappropriate assumptions you might be making.

      Here are some other problems with the code above:

\begin{enumerate}
\item{It mixes definition types between interfaces (e.g.~{\tt List})}
  and implementations (e.g.~{\tt HashMap}). There are very occasionally
  reasons to do this, but this isn't one of them.
\item{It grossly mishandles {\tt tuples}. Were {\tt tuples} a linked}
  list, the function would be quadratic.
\item{{\tt resultIndex} is redundant, misleading, and too long. It}
  takes up a huge amount of space inside the {\tt for} loop.
\item{{\tt tuples[i].word} is repeated three times in quick}
  succession. Stash this into a short variable.
\item{{\tt sortedResult} is too long.}
\item{{\tt iterator} is redundant and too long. The ideal name is}
  {\tt it}.
\item{The {\tt while} loop spends a whole line defining a variable}
  that's almost as long as the expression it replaces.
\item{{\tt quicksort()} is not only dangerous performance-wise, but}
  also not a valid Java function. No Java library function would be
  called this way because Java requires that it be a method call
  (unless you {\tt import static}, but your interviewer might not
  assume you meant that).
\end{enumerate}

      The code also has some technical issues that I don't think are important
      for an interview:

\begin{enumerate}
\item{{\tt Map}s don't support {\tt .add}; the correct method is {\tt}
  .put}. Nobody cares. The compiler would catch this immediately. If
  the interviewer makes a big deal about this, find a company with
  better priorities or more competent interviewers.

\item{{\tt tuples[i]} doesn't work. This is fine; you clearly want to}
  get-by-position. The only thing the interviewer should care about
  here is that this retrieval is linear-time for a {\tt LinkedList},
  and you should know this cold and have a very good excuse for doing
  it this way.

\item{{\tt .size()} could be expensive, but isn't. Your interviewer}
  may not have read the source for {\tt LinkedList}, but there's a
  counter in there that makes {\tt .size()} a constant-time operation.
  It's cool if you can work this into an interview, but odds are that
  if you're counting a linked list you're already doing something
  wrong.
\end{enumerate}

      Here's the right way, with some notational liberties that are appropriate
      during an interview:

\lstset{caption={examples/sorted-index-good.java},name={examples/sorted-index-good.java}}\begin{javacode}
SortedMap<String, List<Integer>> index(final List<Tuple> tuples) {
  final SortedMap<...> result = new TreeMap<>();
  for (final Tuple t : tuples) {
    if (result.containsKey(t.word))
      result.get(t.word).add(t.page);
    else
      result.put(t.word, new ArrayList<Integer>(t.page));
  }
  return result;
} \end{javacode}

      \verb|SortedMap<...>| isn't allowed, but it's really obvious what I mean,
      especially if I explain it as I'm typing. The same thing is true of
      \verb|new ArrayList<Integer>(t.page)|, which actually is valid Java but
      doesn't mean what I want it to (in real life it sets the initial capacity
      of the list). In an interview I would either make a point of mentioning
      this redefinition, or would just write out the right code.

      It's worth mentioning that I'm being a little more pedantic than
      necessary about a few things. In particular, an interviewer is unlikely
      to care much whether you use {\tt containsKey} vs {\tt contains} and
      stuff like that. Knowing these details adds a bit to your credibility,
      and it's something I try to do, but it won't ever be the difference
      between passing and failing an interview.

\subsection{High-level design questions}
    These are usually problems for which you don't write code, but instead just
    describe how you would design a solution. Failure and success are more
    nebulous here, but generally there are two ways to fail one of these
    questions:

\begin{enumerate}
\item{The problem has a well-known solution and you completely miss it.}
\item{You fail to convince the interviewer that you could actually solve}
  the problem given enough time.
\end{enumerate}

    An example of the first failure mode would be something like, ``design a
    data structure that lets me get the minimum and maximum elements in $O(log
    n)$ time'' and you don't think of either a sorted list or a binary tree.
    You might be able to recover if you invent either one, but it's better (and
    far more reassuring to the interviewer) if you just know how they work.

    The second failure mode happens either when you're completely out of your
    depth, or when you don't adopt the ``give up or die trying'' mentality
    thoroughly enough. If you're good at interviewing, neither should happen to
    you, and here's why. Being out of your depth isn't a dealbreaker if you're
    sufficiently resourceful and/or curious. If you don't have a clue, start
    asking definitional questions until the problem makes a little more sense.
    For example:

\begin{quote}
\em
How many bits of entropy are in each word in English?

``Can I ask a question?'' \\
``Sure, go for it.'' \\
``What is entropy?'' \\
``Entropy is information content, measured as Shannon Entropy.'' \\
``Do you mind if I look this up on Wikipedia?''
\end{quote}

    You'll probably get mixed responses to the Wikipedia question. But it's a
    good question to ask because it shows that you know how to teach yourself
    to solve problems you can't yet work with. If the interviewer says yes you
    should be all set, but let's suppose they say no:

\begin{quote}
\em
``Nope, you've got to solve this problem with no external resources.'' \\
``Ok, I think I can solve this problem but not with just what I know. Can
  you relate entropy to other units of measure?'' \\
``I guess you can think of entropy as the log of probability.'' \\
``Aha, so we're looking at words in terms of their probability
  distribution.'' \\
``That's right.'' \\
``Ok, so I guess more common words are less informative because they
  appear more often. Does this mean they have lower entropy?''
\end{quote}

    This is naive and misguided question, but it forces the interviewer to
    either fix your thinking or be pathologically unhelpful. This is exactly
    the situation you want to create when you're operating at a disadvantage.
    Here's one way the conversation can proceed:

\begin{quote}
\em
``Well, a word by itself doesn't have entropy; it's a property of the
  probability distribution.'' \\
``Interesting, so we don't know the information content of a single
  word.'' \\
``That's right, because you need to consider its context to know how
  common it is.'' \\
``Ok, so entropy must have something to do with how the probabilities
  themselves are distributed.''
\end{quote}

    And so on. When the interviewer is your only source of information, it's
    critical to keep the conversation going and maintain enough curiosity to
    find your way around the problem space. Interviews are conversations as
    much as they are anything else, and how you think is as important as the
    knowledge you bring to the table.

    It's worth noting that in order for this to work, you need to have a fairly
    broad base of knowledge to tie things to. The interviewer has neither the
    time nor inclination to teach you computer science or math from the ground
    up, so you have to be making some intelligent connections for them to be
    remotely confident that you'll work out.

\subsubsection{IO scheduler}
\begin{quote}
\em
You're writing an operating system and inventing your own IO scheduling
strategy. How would you design it?
\end{quote}

      This is a really open-ended question, but you'll get those sometimes.
      Here's how a good interview might go:

\begin{quote}
\em
``Ok, so we're scheduling programs' requests to make IO?'' \\
``That's right.'' \\
``Ok. These programs can read and write, and let's treat each request
  as having an offset and length from the device.'' \\
``Yep, you can assume that.'' \\
``I guess each program should have its own IO priority. So stuff like
  databases can get first access when it needs to.'' \\
``Can you see anything that might go wrong if you do it this way?'' \\
``Well, theoretically we might swap to disk, and that's probably being
  scheduled too. But then, maybe we just say that it's inadvisable to
  deprioritize the swapper.'' \\
``Fair enough.'' \\
``Ok, so every process is going through the same scheduler. I guess the
  question now is what to do at any given moment. Let's keep a priority
  queue of requests.'' \\
``How do you sort the queue?'' \\
``Program priority.... oh, and we also need to prevent starvation I
  guess. There's probably a standard way of dealing with this.'' \\
``There is, but suppose you're inventing it from scratch.'' \\
``Ok, well one way to do it is to keep track of the request times. As
  requests get older, they get bumped up the queue even if they're
  technically lower priority.'' \\
``That sounds reasonable. What would that sorting function look like?''
\\
``I guess it would be something like $priority + k(delay)$?'' \\
``How might that function not do what you want?'' \\
``I'm not sure. I guess if $k$ is really small it would still allow
  starvation. But older requests will still outweigh new ones
  eventually.'' \\
``Stepping back, what are some other considerations you might want to
  make in your scheduler?'' \\
``Let's see...~we're handling priority correctly, so I'm not sure.''
``Ok, here's a hint. Suppose you've got two programs running at the
  same priority. One of them is writing 10GB in 4MB chunks, and the
  other one is writing all over the place in tiny 4KB chunks. What's
  going to happen here?'' \\
``Oh, that's awful. The random writer is going to introduce all kinds
  of seek delays into the sequential one. I should have thought of
  that.'' \\
``Yep, exactly. What do you think is ideal behavior in this scenario?''
\\
``I'm not quite sure. Intuitively it seems like we should let the
  sequential one run uninterrupted for a while, but then what if you
  have two such processes? At that point you're balancing sequential
  access against delays.'' \\
``Right, it's a tradeoff between latency and throughput. Is that a
  problem?'' \\
``I guess it isn't; that could be a system-wide preference.'' \\
``Right. So any ideas about the sequential process?''
\end{quote}

      And so forth. Even though the interviewer had to prompt and give hints,
      it was mostly just to direct the conversation, not because the candidate
      was messing stuff up. Some stuff the candidate did right:

\begin{itemize}
\item{Scoped out the problem before trying to solve it. The problem as}
  stated actually wasn't solvable because it was underspecified, but
  the candidate asked enough questions to get something tractable.

\item{Recognized that real-world constraints applied to the problem}
  and used them (e.g.~mentioning disk swapping). Not all problems are
  amenable to this kind of reasoning, but this one is. An interviewer
  will usually appreciate this kind of perspective because junior
  developers often don't have it.

\item{Pointed out that the problem was most likely solved already.}
  This is a good thing to do because the interviewer might immediately
  cut over to asking about established solutions. If you don't offer an
  existing solution, an interviewer might suspect you reinvent the
  wheel too eagerly.

\item{Described the solution at the same level of detail as the}
  interviewer constrained the problem. The candidate didn't burst into
  code or mention direct memory access. Either of these would be
  addressed at a different level of problem, and the candidate's job is
  to recognize this.

\item{Offered some relevant technical details. Mentioning a priority}
  queue isn't out of place at this point because priority queues are
  ideas, not data structures. Had the candidate mentioned using a
  minheap, the interviewer would have had leverage to change the
  problem description in a way that would require the candidate to back
  out of that recommendation (maybe the sorting criterion changes over
  time, so you'd have to re-heapify everything). This wastes time and
  suggests that the candidate might be prematurely optimizing or
  otherwise operating at the wrong level of abstraction.

\item{Remained willing to change the solution's design. Because the}
  problem isn't written down anywhere, it's going to change constantly
  until the interview ends. It's important to treat it this way and not
  commit to anything too low-level. You want to remain as agile as the
  interviewer.
\end{itemize}

      I tried to write an example of a bad interview, but I couldn't figure out
      how the conversation would actually go. A lot of times a bad interview is
      just someone who gets very quiet, or who doesn't understand the
      interviewer and doesn't then ask questions to fix it. A curious person
      who's sufficiently out of the loop might still fail a question like this,
      but odds of success go way up if you stay engaged.

\subsection{Practice problem solutions}\label{sec:practice-problem-solutions}
\begin{enumerate}
\item{Return true if palindrome.}

\lstset{caption={solutions/palindrome.java},name={solutions/palindrome.java}}\begin{javacode}
boolean palindrome(final String s) {
  for (int i = 0, j = s.length() - 1; i < j; ++i, --j)
    if (s.charAt(i) != s.charAt(j))
      return false;
  return true;
} \end{javacode}

\item{Return true if geometric series.}

\lstset{caption={solutions/geometric.java},name={solutions/geometric.java}}\begin{javacode}
boolean geometric(final double[] xs) {
  if (xs.length < 2)
    return true;
  if (xs[0] == 0) {
    for (int i = 0; i < xs.length; ++i)
      if (xs[i] != 0)
        return false;
    return true;
  }

  final double ratio = xs[1] / xs[0];
  for (int i = 2; i < xs.length; ++i)
    if (Math.abs(xs[i] / xs[i - 1] - ratio) > EPSILON)
      return false;
  return true;
} \end{javacode}

\item{Return the word count of a string without allocating memory.}

\lstset{caption={solutions/wordcount.java},name={solutions/wordcount.java}}\begin{javacode}
int wordcount(final String s) {
  boolean inWord = false;
  int     words  = 0;
  for (int i = 0; i < s.length(); ++i) {
    final char c = s.charAt(i);
    if (c >= 'A' && c <= 'Z' || c >= 'a' && c <= 'z') {
      if (!inWord)
        ++words;
      inWord = true;
    } else
      inWord = false;
  }
  return words;
} \end{javacode}
\end{enumerate}

\end{document}
